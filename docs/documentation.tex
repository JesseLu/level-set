\documentclass{article}
\usepackage{amsmath}

\newcommand{\be}{\begin{equation}}
\newcommand{\ee}{\end{equation}}
\newcommand{\grad}{\nabla}
\newcommand{\ext}{\text{ext}}

\title{Documentation for Level Set Method Package}
\author{Jesse Lu, \texttt{jesselu@stanford.edu}}

\begin{document}
\maketitle
\tableofcontents

\section{What is a level set?}
A level set is an \emph{implicit} definition of a boundary. 
For example, in two dimensions, 
    \be \phi(x,y) = x^2 + y^2 - 1 \ee
can \emph{implicitly} define a circle of radius 1 as the boundary
    \be \phi(x,y) = 0, \ee
which is the 0-level set of $\phi$.

We will use the convention of specifying boundaries using the 0-level set of $\phi$. 
Also, we consider the interior of a boundary as $\phi<0$ and its exterior as $\phi>0$.


\section{Basic mathematical properties}
\subsection{Derivative}
We will use the notation $\phi_x= \delta \phi / \delta x$ to signify the partial derivative.

On a discretized grid, the derivative can be calculated either as
    \be \phi_x^+(i,j) = \frac{\phi(i+1,j) - \phi(i,j)}{\Delta x}, \ee
    \be \phi_x^0(i,j) = \frac{\phi(i+1,j) - \phi(i-1,j)}{2 \Delta x}, \ee
    or
    \be \phi_x^-(i,j) = \frac{\phi(i,j) - \phi(i-1,j)}{\Delta x}, \ee
the appropriate choice usually given by stability and accuracy considerations.

Similarily, the second derivative can be calculated as 
    \be \phi_{xx}(i,j) = \frac{\phi(i-1,j) - 2\phi(i,j) + \phi(i-1,j)}
        {\Delta x^2}. \ee

\subsection{Gradient}
The gradient of $\phi$ is 
    \be \grad\phi = (\phi_x, \phi_y, \phi_z), \ee
where the appropriate partials are assumed.

Note that the outward (unit) normal of the boundary is then given by
    \be \vec{N} = \frac{\grad\phi}{|\grad\phi|} \ee
along the boundary.

\subsection{Curvature}
The curvature is defined as
    \be \kappa = \grad \cdot \vec{N} = 
        \frac{\phi_x^2\phi_{yy} - 2\phi_x\phi_y\phi_{xy} + \phi_y^2\phi_{xx}}
        {|\grad\phi|^3}. \ee


\section{Signed distance function}
A signed distance function $\phi$ not only defines a boundary on its 0-level set but has the additional property,
    \be |\grad\phi| = 1, \ee
which allows movement of the boundary to be ``well-defined''.

To construct the signed distance function, find the stable solution of 
    \be \phi_t + S(\phi_0)(|\grad\phi - 1) = 0 \ee
where
    \be S(\phi_0) = \frac{\phi_0}{\sqrt{\phi_0^2 - \Delta x^2}} \ee
and $\phi_0$ is the initial description of the interface. See chapter 7 of ref.~\cite{OF} for more details. Note that this requires moving an interface relative to its normal direction, covered below.

\section{Interface motion using a Hamilton-Jacobi formulation}
In order to add time dynamics (motion) to the interface, we cast $\phi$ into the Hamilton-Jacobi equation
\be \phi_t + H(\grad \phi) = 0. \ee

\subsection{Numerical discretization in time}
We use the forward Euler time discretization, following Eq. 5.9 of ref.~\cite{OF},
\be \frac{\phi^{n+1} - \phi^n}{\Delta t} + H^n(\grad \phi) = 0. \ee

\subsection{Numerical discretization in space}
We employ Godunov's scheme as outline in sections 5.3.3 and 6.2--6.4 in ref.~\cite{OF} to choose the most appropriate spatial derivative of $\phi$. 
The scheme is based on the following mechanism,
\be \hat{H} = \ext_x \ext_y H(\phi_x, \phi_y), \ee
where
\be \ext_x = \begin{cases}
    \min_{\phi_x \in I_x} H(\phi_x, \phi_y)& \text{if $\phi_x^- < \phi_x^+$,} \\
    \max_{\phi_x \in I_x} H(\phi_x, \phi_y)& \text{if $\phi_x^- > \phi_x^+$.}
    \end{cases} \ee
Here, $I_x$ denotes the interval $[\phi_x^-, \phi_x^+]$.


\section{Motion in an external velocity field}
If $H(\grad \phi) = \vec{V} \cdot \grad \phi = u\phi_x + v\phi_y$, then Godunov's scheme chooses the derivatives in the following way,
\be \phi_x = \begin{cases}
    \phi_x^- & \text{if $u > 0$,} \\
    \phi_x^+ & \text{if $u < 0$.} \end{cases} \ee

Also, the time interval should be 
\be \Delta t = \frac{\alpha} 
    {\max\left\{\frac{|u|}{\Delta x} + \frac{|v|}{\Delta y} \right\}} \ee
in order to obey the Courant-Friedreichs-Lewy (CFL) stability condition, with $\alpha < 1$. A good choice is apparently $\alpha = 0.9$.

\section{Motion in the normal direction}
If $H(\grad \phi) = a |\grad \phi|$, then Godunov's scheme is somewhat more complicated, but can be summarized as,

\begin{center}
\begin{tabular}{c|c c c}
Choice of $\phi_x$ & sign of $\phi_x^-$ & sign of $\phi_x^+$ & Type \\ \hline
$\phi_x^-$ & positive & positive & upwinding \\
$\phi_x^+$ & negative & negative & upwinding \\
$0$ & negative & positive & expansion \\
$\text{argmax}_{\phi_x^\pm}|a\phi_x^\pm|$ & positive & negative & shock 
\end{tabular}
\end{center}

Alternatively, Eqs. 6.3 and 6.4 from ref.~\cite{OF} can be used, which state
\be \phi_x^2 = \begin{cases}
    \max(\max(\phi_x^-,0)^2, \min(\phi_x^+,0)^2) & \text{if $a > 0$,} \\
    \max(\min(\phi_x^-,0)^2, \max(\phi_x^+,0)^2) & \text{if $a < 0$.} 
    \end{cases} \ee

A stable time step for $\alpha < 1$ is 
\be \Delta t = \frac{\alpha}{\max\left\{\frac{|H_x|}
    {\Delta x} + \frac{|H_y|}{\Delta y}\right\}}, \ee
as expected from the CFL condition, where $H_x = a \phi_x^2 / |\grad \phi|$.

\section{Motion involving mean curvature}
...

\begin{thebibliography}{99}
\bibitem{OF} Stanley Osher, Ronald Fedkiw, \emph{Level Set Methods and Dynamic Implicit Surfaces} (Springer 2003).
\end{thebibliography}
\end{document}
